\documentclass[13pt, a4paper]{article}
\usepackage[utf8]{inputenc}
\usepackage{fontenc}
\usepackage{xcolor}
\usepackage{hyperref}
\usepackage[italian]{babel}
\usepackage[inline]{enumitem}
\usepackage{graphicx}
\usepackage{cleveref}

\setlist[enumerate,1]{label=\arabic*}
\setlist[enumerate,2]{label=\theenumi.\arabic*}
\setlist[enumerate,3]{label=\theenumii.\arabic*}

\graphicspath{ {res/} }
\usepackage{url}
\usepackage{acronym}
%\usepackage{natbib}
\usepackage{makeidx}
\usepackage{listings}
\usepackage[square,numbers,sort]{natbib}
\usepackage{tabularx}
\usepackage{float}

% version
\newcommand{\versionmajor}{0}
\newcommand{\versionminor}{1}
\newcommand{\versionpatch}{2}
\newcommand{\version}{\versionmajor.\versionminor.\versionpatch}
\newenvironment{inlinelist}{\begin{enumerate*}[label=\emph{(\roman*)}]}{\end{enumerate*}}
\typeout{Document version: \version}

% acronyms
\acrodef{FC}{Field Calculus}
\acrodef{AC}{Aggregate Computing}
\acrodef{dsl}[DSL]{Domain Specific Language}
\newcommand{\ck}{\emph{Collektive}}

\title{\LARGE
    Un approccio unificante alla programmazione di dispositivi eterogenei nell'edge-cloud continuum \\ \small Prima relazione trimestrale di progetto
}

\author{
   Borsista: \\Angela Cortecchia \\ \small angela.cortecchia@unibo.it
    \and
    Tutor accademico: \\Danilo Pianini \\ \small danilo.pianini@unibo.it
}

%\date{\small Academic year}
%\makeindex

\begin{document}
\maketitle
\clearpage

%%%%%%%%%%%%%%%%%%%%body%%%%%%%%%%%%%%%%%%%%

\section{Recap del contesto}
\label{sec:context}
In un contesto in cui sono sempre in maggior numero i dispositivi dotati di capacità computazionali,
che presentano la necessità di un'organizzazione e di un coordinamento tra loro,
è necessario trovare una soluzione uniforme che permetta di programmare il comportamento e la coordinazione dei dispositivi.

Questi dispositivi possono essere di natura eterogenea, ovvero possono avere architetture e sistemi operativi diversi,
e possono essere distribuiti in un'area geografica più o meno estesa.
%
La diversa natura di questi dispositivi può portare a problemi di interoperabilità e di comunicazione tra di essi.
%
Possono trovare applicazioni in vari contesti diversi, come ad esempio \emph{smart cities}, \emph{swarm robotics} e \emph{crowd management},
includendo dispositivi di tipo wearable come smartwatch e smartphone.

La gestione del singolo dispositivo in questi contesti non è banale, possono ad esempio essere presenti 200 o più dispositivi
connessi tra loro all'interno di una rete.
%
Gestirli tutti individualmente porterebbe a problemi di scalabilità, manutenibilità ed etereogeneità.

Tra le varie soluzioni proposte troviamo l'uso del Cloud Computing o dell'Edge e Fog Computing,
ma entrambe presentano dei problemi di scalabilità e di latenza o non sfruttano appieno le oppurtinità dell'infrastruttura.

Per questi motivi si vuole passare dall'approccio classico con device centrico, ovvero dove il programma è scritto per il singolo dispositivo,
ad un approccio collettivo basato sulla programmazione macroscopica del comportamento dei dispositivi,
usando tecniche di \emph{\ac{AC}}.

\ac{AC} è un paradigma di programmazione che permette di definire un unico comportamento di un insieme di dispositivi,
definendo l'interazione fra loro.
%
È fondato sull'astrazione del \ac{FC} che consente di esprimere il comportamento auto-organizzante di reti
di dispositivi come funzioni riutilizzabili che operano sui campi, dette ``costrutti'' o ``blocchi''.

Per fare ciò, viene sviluppato un \ac{dsl}, denominato \ck{}, basato sui concetti di \ac{AC} e \ac{FC} ed implementato in Kotlin Mulitplatform,
che permette di scrivere codice in Kotlin e di compilarlo per diverse piattaforme, come JVM, JS, Native, iOS e Android.

\section{Lavoro svolto}\label{sec:lavoro-svolto}

Per prima cosa è servito capire quali fossero effettivamente i costrutti necessari da inserire all'interno del DSL di \ck{}.




\label{sec:contribution}
\begin{figure}
    \centering
    \includegraphics[width=\textwidth]{images/collektive_timeline}
    \caption{Timeline del progetto proposta.}
    \label{fig:timeline}
\end{figure}

\section{Obiettivi raggiunti}\label{sec:obiettivi-raggiunti}

\section{Prossimi obiettivi}\label{sec:prossimi-obiettivi}




%%%%%%%%%%%%%%%%%%%%end%%%%%%%%%%%%%%%%%%%%

% add more ....

\cite[none]{none}
\bibliographystyle{plain}
\bibliography{bibliography}

\end{document}